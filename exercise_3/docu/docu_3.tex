_{•}%----------------------------------------------------------------------------------------
% PACKAGES AND OTHER DOCUMENT CONFIGURATIONS
%----------------------------------------------------------------------------------------
\documentclass{article}
\usepackage{fancyhdr} % Required for custom headers
\usepackage{lastpage} % Required to determine the last page for the footer
\usepackage{extramarks} % Required for headers and footers
\usepackage[usenames,dvipsnames]{color} % Required for custom colors
\usepackage{graphicx} % Required to insert images
\usepackage{listings} % Required for insertion of code
\usepackage{courier} % Required for the courier font
\usepackage{lipsum} % Used for inserting dummy 'Lorem ipsum' text into the template
\usepackage[utf8]{inputenc}
\usepackage[ngerman]{babel}
% Margins
\topmargin=-0.45in
\evensidemargin=0in
\oddsidemargin=0in
\textwidth=6.5in
\textheight=9.0in
\headsep=0.25in
\linespread{1.1} % Line spacing
% Set up the header and footer
\pagestyle{fancy}
%\lhead{\hmwkAuthorName} % Top left header
\chead{\hmwkClass\ : \hmwkTitle} % Top center head
\rhead{\firstxmark} % Top right header
\lfoot{\lastxmark} % Bottom left footer
\cfoot{} % Bottom center footer
\rfoot{Page\ \thepage\ of\ \protect\pageref{LastPage}} % Bottom right footer
\renewcommand\headrulewidth{0.4pt} % Size of the header rule
\renewcommand\footrulewidth{0.4pt} % Size of the footer rule
\setlength\parindent{0pt} % Removes all indentation from paragraphs
%----------------------------------------------------------------------------------------
% CODE INCLUSION CONFIGURATION
%----------------------------------------------------------------------------------------
\definecolor{MyDarkGreen}{rgb}{0.0,0.4,0.0} % This is the color used for comments
\lstloadlanguages{Perl} % Load Perl syntax for listings, for a list of other languages supported see: ftp://ftp.tex.ac.uk/tex-archive/macros/latex/contrib/listings/listings.pdf
\lstset{language=Perl, % Use Perl in this example
frame=single, % Single frame around code
basicstyle=\small\ttfamily, % Use small true type font
keywordstyle=[1]\color{Blue}\bf, % Perl functions bold and blue
keywordstyle=[2]\color{Purple}, % Perl function arguments purple
keywordstyle=[3]\color{Blue}\underbar, % Custom functions underlined and blue
identifierstyle=, % Nothing special about identifiers
commentstyle=\usefont{T1}{pcr}{m}{sl}\color{MyDarkGreen}\small, % Comments small dark green courier font
stringstyle=\color{Purple}, % Strings are purple
showstringspaces=false, % Don't put marks in string spaces
tabsize=5, % 5 spaces per tab
%
% Put standard Perl functions not included in the default language here
morekeywords={rand},
%
% Put Perl function parameters here
morekeywords=[2]{on, off, interp},
%
% Put user defined functions here
morekeywords=[3]{test},
%
morecomment=[l][\color{Blue}]{...}, % Line continuation (...) like blue comment
numbers=left, % Line numbers on left
firstnumber=1, % Line numbers start with line 1
numberstyle=\tiny\color{Blue}, % Line numbers are blue and small
stepnumber=5 % Line numbers go in steps of 5
}
% Creates a new command to include a perl script, the first parameter is the filename of the script (without .pl), the second parameter is the caption
\newcommand{\perlscript}[2]{
\begin{itemize}
\item[]\lstinputlisting[caption=#2,label=#1]{#1.pl}
\end{itemize}
}
%----------------------------------------------------------------------------------------
% DOCUMENT STRUCTURE COMMANDS
% Skip this unless you know what you're doing
%----------------------------------------------------------------------------------------
% Header and footer for when a page split occurs within a problem environment
\newcommand{\enterProblemHeader}[1]{
%\nobreak\extramarks{#1}{#1 continued on next page\ldots}\nobreak
%\nobreak\extramarks{#1 (continued)}{#1 continued on next page\ldots}\nobreak
}
% Header and footer for when a page split occurs between problem environments
\newcommand{\exitProblemHeader}[1]{
%\nobreak\extramarks{#1 (continued)}{#1 continued on next page\ldots}\nobreak
%\nobreak\extramarks{#1}{}\nobreak
}
\setcounter{secnumdepth}{0} % Removes default section numbers
\newcounter{homeworkProblemCounter} % Creates a counter to keep track of the number of problems
\newcommand{\homeworkProblemName}{}
\newenvironment{homeworkProblem}[1][Problem \arabic{homeworkProblemCounter}]{ % Makes a new environment called homeworkProblem which takes 1 argument (custom name) but the default is "Problem #"
\stepcounter{homeworkProblemCounter} % Increase counter for number of problems
\renewcommand{\homeworkProblemName}{#1} % Assign \homeworkProblemName the name of the problem
\section{\homeworkProblemName} % Make a section in the document with the custom problem count
%\enterProblemHeader{\homeworkProblemName} % Header and footer within the environment
}{
%\exitProblemHeader{\homeworkProblemName} % Header and footer after the environment
}
\newcommand{\problemAnswer}[1]{ % Defines the problem answer command with the content as the only argument
\noindent\framebox[\columnwidth][c]{\begin{minipage}{0.98\columnwidth}#1\end{minipage}} % Makes the box around the problem answer and puts the content inside
}
\newcommand{\homeworkSectionName}{}
\newenvironment{homeworkSection}[1]{ % New environment for sections within homework problems, takes 1 argument - the name of the section
\renewcommand{\homeworkSectionName}{#1} % Assign \homeworkSectionName to the name of the section from the environment argument
\subsection{\homeworkSectionName} % Make a subsection with the custom name of the subsection
%\enterProblemHeader{\homeworkProblemName\ [\homeworkSectionName]} % Header and footer within the environment
}{
%\enterProblemHeader{\homeworkProblemName} % Header and footer after the environment
}
%----------------------------------------------------------------------------------------
% NAME AND CLASS SECTION
%----------------------------------------------------------------------------------------
\newcommand{\hmwkTitle}{Übung\ \#3} % Assignment title
\newcommand{\hmwkDueDate}{Mittwoch,\ 18.\ November\ 2014} % Due date
\newcommand{\hmwkClass}{Parallel Computer Architecture} % Course/class
\newcommand{\hmwkClassTime}{} % Class/lecture time
\newcommand{\hmwkClassInstructor}{} % Teacher/lecturer
\newcommand{\hmwkAuthorName}{Günther Schindler, Fabian Finkeldey, Shamna Shyju} % Your name
%----------------------------------------------------------------------------------------
% TITLE PAGE
%----------------------------------------------------------------------------------------
\title{
\vspace{2in}
\textmd{\textbf{\hmwkClass:\ \hmwkTitle}}\\
\normalsize\vspace{0.1in}\small{Abgabe\ am\ \hmwkDueDate}\\
\vspace{0.1in}\large{\textit{\hmwkClassTime}}
\vspace{3in}
}
\author{\textbf{\hmwkAuthorName}}
\date{} % Insert date here if you want it to appear below your name
%----------------------------------------------------------------------------------------
\begin{document}
\maketitle
%----------------------------------------------------------------------------------------
% TABLE OF CONTENTS
%----------------------------------------------------------------------------------------
%\setcounter{tocdepth}{1} % Uncomment this line if you don't want subsections listed in the ToC
\newpage
\tableofcontents
\newpage
%----------------------------------------------------------------------------------------
% Relaxation
%----------------------------------------------------------------------------------------
\begin{homeworkProblem}[NBody]
Für die dritte Übung ist ein Programm zu implementieren, welches die Anziehungkräfte und die daraus resultierenden Bewegungen einer beliebigen Anzahl von Massepunkten simuliert.
\\
Ein Massepunkt besitzt also 5 signifikante Werte:\\
Die Geschwindigkeit in X Richtung,\\
Die Geschwindigkeit in Y Richtung,\\
Die Position in X Richtung,\\
Die Position in Y Richtung\\
und die Masse.\\
\\
Um dieses zu Realisierung wurde ein entsprechendes struct implementiert:
\begin{lstlisting}{c}
typedef struct sMassPoint
{
	double dVelocityX;
	double dVelocityY;
	double dPositionX;
	double dPositionY;
	double dMass;
} sMassPoint;
\end{lstlisting}
Diese MassPoints werden mit einer Geschwindigkeit von 0.0 und zufälliger Position so wie Masse initialisiert. Um die Zufallswerte in einem Wertebereich zu halten, der sich zur Anzeige eignet (d.h.: Werte bei denen innerhalb von 100 Iterationen nennenswerte Positionsänderungen zu beobachten sind) und in dem es nicht zu einem Overflow kommt, lässt sich der Maximalwert begrenzen. Ein Maximum von 25.000 hat sich bei den ersten Tests des Programms als sinnvoller Kompromiss herausgestellt.
\\
Der eigentliche Kern des Programms ist die Simulation:
\begin{lstlisting}{c}
sMassPoint* vSimulate(int numberOfMassPoints, int iterations, int timestep, int maxValue)
{
	//Initialize all MassPoints
	sMassPoint* massPoints = sInitMassPoints(numberOfMassPoints, maxValue);
	int n = 0;
	//Do all iterations
	for(n; n<iterations; n++)
	{
		/*
		* Do a pairwise calculation and application of forces
		* and update all velocities with vApplyForces()
		*/
		int i = 0;
		for(i; i<numberOfMassPoints-1; i++)
		{
			int j = i+1;
			for(j; j<numberOfMassPoints; j++)
			{
				vApplyForces(&massPoints[i], &massPoints[j], timestep);
			}
		}
		/*
		* Update Position on all Masspoints after calculating
		* Speeds
		*/
		int k=0;
		for(k; k<numberOfMassPoints; k++)
		{
			vUpdatePosition(&massPoints[k], timestep);
		}
	}
	return massPoints;
}
\end{lstlisting}

Die äußerste for-Schleife representiert die n auszuführenden Iterationen. Die nächst-innere läuft über alle Massepunkte und führt vApplyForces für den aktuellen und alle folgenden Massepunkte aus.\\
vApplyForces berechnet dabei die Kräfte, die zwei gegebene Massepunkte aufeinander ausüben und berechnet daraus (und aus der Masse des jeweiligen Punktes) die Beschleunigung für beide Massepunkte. Mit dem gegebenen Zeitschritt wird die so berechnete Geschwindigkeitsänderung auf die aktuelle Geschwindigkeit beider Massepunkte addiert.\\
Da in diesem Schritt beide Massepunkte manipuliert werden, muss kein paarweiser Vergleich aller Punkte durchgeführt werden. Angenommen es gäbe zwei Punkte, A und B. So ist der Vergleich (A,B) in diesem Fall identisch mit (B,A). Der zweite Schritt muss also nicht ausgeführt werden.\\
Nachdem alle Massepunkte miteinander verglichen wurden und die Geschwindigkeiten aktualisiert wurden, wird der eigentliche Zeitschritt ausgeführt. Basierend auf der gewählten Schrittweite und der momentanen Geschwindigkeit wird nun die Position für alle Massepunkte aktualisiert. Damit ist eine Iteration abgeschlossen.\\

Um die Funktion des Programms zu demonstrieren wird am Ende der Simulation exemplarisch ein Massepunkt mit all seinen Werten ausgegeben. Eine vollständige Ausgabe des Zustands aller Massepunkte zu jeder Iteration ist über eine Kommandozeilen-Ausgabe leider zu unübersichtlich.

\end{homeworkProblem}
\end{document}
